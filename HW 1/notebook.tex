
% Default to the notebook output style

    


% Inherit from the specified cell style.




    
\documentclass[11pt]{article}

    
    
    \usepackage[T1]{fontenc}
    % Nicer default font (+ math font) than Computer Modern for most use cases
    \usepackage{mathpazo}

    % Basic figure setup, for now with no caption control since it's done
    % automatically by Pandoc (which extracts ![](path) syntax from Markdown).
    \usepackage{graphicx}
    % We will generate all images so they have a width \maxwidth. This means
    % that they will get their normal width if they fit onto the page, but
    % are scaled down if they would overflow the margins.
    \makeatletter
    \def\maxwidth{\ifdim\Gin@nat@width>\linewidth\linewidth
    \else\Gin@nat@width\fi}
    \makeatother
    \let\Oldincludegraphics\includegraphics
    % Set max figure width to be 80% of text width, for now hardcoded.
    \renewcommand{\includegraphics}[1]{\Oldincludegraphics[width=.8\maxwidth]{#1}}
    % Ensure that by default, figures have no caption (until we provide a
    % proper Figure object with a Caption API and a way to capture that
    % in the conversion process - todo).
    \usepackage{caption}
    \DeclareCaptionLabelFormat{nolabel}{}
    \captionsetup{labelformat=nolabel}

    \usepackage{adjustbox} % Used to constrain images to a maximum size 
    \usepackage{xcolor} % Allow colors to be defined
    \usepackage{enumerate} % Needed for markdown enumerations to work
    \usepackage{geometry} % Used to adjust the document margins
    \usepackage{amsmath} % Equations
    \usepackage{amssymb} % Equations
    \usepackage{textcomp} % defines textquotesingle
    % Hack from http://tex.stackexchange.com/a/47451/13684:
    \AtBeginDocument{%
        \def\PYZsq{\textquotesingle}% Upright quotes in Pygmentized code
    }
    \usepackage{upquote} % Upright quotes for verbatim code
    \usepackage{eurosym} % defines \euro
    \usepackage[mathletters]{ucs} % Extended unicode (utf-8) support
    \usepackage[utf8x]{inputenc} % Allow utf-8 characters in the tex document
    \usepackage{fancyvrb} % verbatim replacement that allows latex
    \usepackage{grffile} % extends the file name processing of package graphics 
                         % to support a larger range 
    % The hyperref package gives us a pdf with properly built
    % internal navigation ('pdf bookmarks' for the table of contents,
    % internal cross-reference links, web links for URLs, etc.)
    \usepackage{hyperref}
    \usepackage{longtable} % longtable support required by pandoc >1.10
    \usepackage{booktabs}  % table support for pandoc > 1.12.2
    \usepackage[inline]{enumitem} % IRkernel/repr support (it uses the enumerate* environment)
    \usepackage[normalem]{ulem} % ulem is needed to support strikethroughs (\sout)
                                % normalem makes italics be italics, not underlines
    

    
    
    % Colors for the hyperref package
    \definecolor{urlcolor}{rgb}{0,.145,.698}
    \definecolor{linkcolor}{rgb}{.71,0.21,0.01}
    \definecolor{citecolor}{rgb}{.12,.54,.11}

    % ANSI colors
    \definecolor{ansi-black}{HTML}{3E424D}
    \definecolor{ansi-black-intense}{HTML}{282C36}
    \definecolor{ansi-red}{HTML}{E75C58}
    \definecolor{ansi-red-intense}{HTML}{B22B31}
    \definecolor{ansi-green}{HTML}{00A250}
    \definecolor{ansi-green-intense}{HTML}{007427}
    \definecolor{ansi-yellow}{HTML}{DDB62B}
    \definecolor{ansi-yellow-intense}{HTML}{B27D12}
    \definecolor{ansi-blue}{HTML}{208FFB}
    \definecolor{ansi-blue-intense}{HTML}{0065CA}
    \definecolor{ansi-magenta}{HTML}{D160C4}
    \definecolor{ansi-magenta-intense}{HTML}{A03196}
    \definecolor{ansi-cyan}{HTML}{60C6C8}
    \definecolor{ansi-cyan-intense}{HTML}{258F8F}
    \definecolor{ansi-white}{HTML}{C5C1B4}
    \definecolor{ansi-white-intense}{HTML}{A1A6B2}

    % commands and environments needed by pandoc snippets
    % extracted from the output of `pandoc -s`
    \providecommand{\tightlist}{%
      \setlength{\itemsep}{0pt}\setlength{\parskip}{0pt}}
    \DefineVerbatimEnvironment{Highlighting}{Verbatim}{commandchars=\\\{\}}
    % Add ',fontsize=\small' for more characters per line
    \newenvironment{Shaded}{}{}
    \newcommand{\KeywordTok}[1]{\textcolor[rgb]{0.00,0.44,0.13}{\textbf{{#1}}}}
    \newcommand{\DataTypeTok}[1]{\textcolor[rgb]{0.56,0.13,0.00}{{#1}}}
    \newcommand{\DecValTok}[1]{\textcolor[rgb]{0.25,0.63,0.44}{{#1}}}
    \newcommand{\BaseNTok}[1]{\textcolor[rgb]{0.25,0.63,0.44}{{#1}}}
    \newcommand{\FloatTok}[1]{\textcolor[rgb]{0.25,0.63,0.44}{{#1}}}
    \newcommand{\CharTok}[1]{\textcolor[rgb]{0.25,0.44,0.63}{{#1}}}
    \newcommand{\StringTok}[1]{\textcolor[rgb]{0.25,0.44,0.63}{{#1}}}
    \newcommand{\CommentTok}[1]{\textcolor[rgb]{0.38,0.63,0.69}{\textit{{#1}}}}
    \newcommand{\OtherTok}[1]{\textcolor[rgb]{0.00,0.44,0.13}{{#1}}}
    \newcommand{\AlertTok}[1]{\textcolor[rgb]{1.00,0.00,0.00}{\textbf{{#1}}}}
    \newcommand{\FunctionTok}[1]{\textcolor[rgb]{0.02,0.16,0.49}{{#1}}}
    \newcommand{\RegionMarkerTok}[1]{{#1}}
    \newcommand{\ErrorTok}[1]{\textcolor[rgb]{1.00,0.00,0.00}{\textbf{{#1}}}}
    \newcommand{\NormalTok}[1]{{#1}}
    
    % Additional commands for more recent versions of Pandoc
    \newcommand{\ConstantTok}[1]{\textcolor[rgb]{0.53,0.00,0.00}{{#1}}}
    \newcommand{\SpecialCharTok}[1]{\textcolor[rgb]{0.25,0.44,0.63}{{#1}}}
    \newcommand{\VerbatimStringTok}[1]{\textcolor[rgb]{0.25,0.44,0.63}{{#1}}}
    \newcommand{\SpecialStringTok}[1]{\textcolor[rgb]{0.73,0.40,0.53}{{#1}}}
    \newcommand{\ImportTok}[1]{{#1}}
    \newcommand{\DocumentationTok}[1]{\textcolor[rgb]{0.73,0.13,0.13}{\textit{{#1}}}}
    \newcommand{\AnnotationTok}[1]{\textcolor[rgb]{0.38,0.63,0.69}{\textbf{\textit{{#1}}}}}
    \newcommand{\CommentVarTok}[1]{\textcolor[rgb]{0.38,0.63,0.69}{\textbf{\textit{{#1}}}}}
    \newcommand{\VariableTok}[1]{\textcolor[rgb]{0.10,0.09,0.49}{{#1}}}
    \newcommand{\ControlFlowTok}[1]{\textcolor[rgb]{0.00,0.44,0.13}{\textbf{{#1}}}}
    \newcommand{\OperatorTok}[1]{\textcolor[rgb]{0.40,0.40,0.40}{{#1}}}
    \newcommand{\BuiltInTok}[1]{{#1}}
    \newcommand{\ExtensionTok}[1]{{#1}}
    \newcommand{\PreprocessorTok}[1]{\textcolor[rgb]{0.74,0.48,0.00}{{#1}}}
    \newcommand{\AttributeTok}[1]{\textcolor[rgb]{0.49,0.56,0.16}{{#1}}}
    \newcommand{\InformationTok}[1]{\textcolor[rgb]{0.38,0.63,0.69}{\textbf{\textit{{#1}}}}}
    \newcommand{\WarningTok}[1]{\textcolor[rgb]{0.38,0.63,0.69}{\textbf{\textit{{#1}}}}}
    
    
    % Define a nice break command that doesn't care if a line doesn't already
    % exist.
    \def\br{\hspace*{\fill} \\* }
    % Math Jax compatability definitions
    \def\gt{>}
    \def\lt{<}
    % Document parameters
    \title{HW1-F18-MIS382N}
    
    
    

    % Pygments definitions
    
\makeatletter
\def\PY@reset{\let\PY@it=\relax \let\PY@bf=\relax%
    \let\PY@ul=\relax \let\PY@tc=\relax%
    \let\PY@bc=\relax \let\PY@ff=\relax}
\def\PY@tok#1{\csname PY@tok@#1\endcsname}
\def\PY@toks#1+{\ifx\relax#1\empty\else%
    \PY@tok{#1}\expandafter\PY@toks\fi}
\def\PY@do#1{\PY@bc{\PY@tc{\PY@ul{%
    \PY@it{\PY@bf{\PY@ff{#1}}}}}}}
\def\PY#1#2{\PY@reset\PY@toks#1+\relax+\PY@do{#2}}

\expandafter\def\csname PY@tok@w\endcsname{\def\PY@tc##1{\textcolor[rgb]{0.73,0.73,0.73}{##1}}}
\expandafter\def\csname PY@tok@c\endcsname{\let\PY@it=\textit\def\PY@tc##1{\textcolor[rgb]{0.25,0.50,0.50}{##1}}}
\expandafter\def\csname PY@tok@cp\endcsname{\def\PY@tc##1{\textcolor[rgb]{0.74,0.48,0.00}{##1}}}
\expandafter\def\csname PY@tok@k\endcsname{\let\PY@bf=\textbf\def\PY@tc##1{\textcolor[rgb]{0.00,0.50,0.00}{##1}}}
\expandafter\def\csname PY@tok@kp\endcsname{\def\PY@tc##1{\textcolor[rgb]{0.00,0.50,0.00}{##1}}}
\expandafter\def\csname PY@tok@kt\endcsname{\def\PY@tc##1{\textcolor[rgb]{0.69,0.00,0.25}{##1}}}
\expandafter\def\csname PY@tok@o\endcsname{\def\PY@tc##1{\textcolor[rgb]{0.40,0.40,0.40}{##1}}}
\expandafter\def\csname PY@tok@ow\endcsname{\let\PY@bf=\textbf\def\PY@tc##1{\textcolor[rgb]{0.67,0.13,1.00}{##1}}}
\expandafter\def\csname PY@tok@nb\endcsname{\def\PY@tc##1{\textcolor[rgb]{0.00,0.50,0.00}{##1}}}
\expandafter\def\csname PY@tok@nf\endcsname{\def\PY@tc##1{\textcolor[rgb]{0.00,0.00,1.00}{##1}}}
\expandafter\def\csname PY@tok@nc\endcsname{\let\PY@bf=\textbf\def\PY@tc##1{\textcolor[rgb]{0.00,0.00,1.00}{##1}}}
\expandafter\def\csname PY@tok@nn\endcsname{\let\PY@bf=\textbf\def\PY@tc##1{\textcolor[rgb]{0.00,0.00,1.00}{##1}}}
\expandafter\def\csname PY@tok@ne\endcsname{\let\PY@bf=\textbf\def\PY@tc##1{\textcolor[rgb]{0.82,0.25,0.23}{##1}}}
\expandafter\def\csname PY@tok@nv\endcsname{\def\PY@tc##1{\textcolor[rgb]{0.10,0.09,0.49}{##1}}}
\expandafter\def\csname PY@tok@no\endcsname{\def\PY@tc##1{\textcolor[rgb]{0.53,0.00,0.00}{##1}}}
\expandafter\def\csname PY@tok@nl\endcsname{\def\PY@tc##1{\textcolor[rgb]{0.63,0.63,0.00}{##1}}}
\expandafter\def\csname PY@tok@ni\endcsname{\let\PY@bf=\textbf\def\PY@tc##1{\textcolor[rgb]{0.60,0.60,0.60}{##1}}}
\expandafter\def\csname PY@tok@na\endcsname{\def\PY@tc##1{\textcolor[rgb]{0.49,0.56,0.16}{##1}}}
\expandafter\def\csname PY@tok@nt\endcsname{\let\PY@bf=\textbf\def\PY@tc##1{\textcolor[rgb]{0.00,0.50,0.00}{##1}}}
\expandafter\def\csname PY@tok@nd\endcsname{\def\PY@tc##1{\textcolor[rgb]{0.67,0.13,1.00}{##1}}}
\expandafter\def\csname PY@tok@s\endcsname{\def\PY@tc##1{\textcolor[rgb]{0.73,0.13,0.13}{##1}}}
\expandafter\def\csname PY@tok@sd\endcsname{\let\PY@it=\textit\def\PY@tc##1{\textcolor[rgb]{0.73,0.13,0.13}{##1}}}
\expandafter\def\csname PY@tok@si\endcsname{\let\PY@bf=\textbf\def\PY@tc##1{\textcolor[rgb]{0.73,0.40,0.53}{##1}}}
\expandafter\def\csname PY@tok@se\endcsname{\let\PY@bf=\textbf\def\PY@tc##1{\textcolor[rgb]{0.73,0.40,0.13}{##1}}}
\expandafter\def\csname PY@tok@sr\endcsname{\def\PY@tc##1{\textcolor[rgb]{0.73,0.40,0.53}{##1}}}
\expandafter\def\csname PY@tok@ss\endcsname{\def\PY@tc##1{\textcolor[rgb]{0.10,0.09,0.49}{##1}}}
\expandafter\def\csname PY@tok@sx\endcsname{\def\PY@tc##1{\textcolor[rgb]{0.00,0.50,0.00}{##1}}}
\expandafter\def\csname PY@tok@m\endcsname{\def\PY@tc##1{\textcolor[rgb]{0.40,0.40,0.40}{##1}}}
\expandafter\def\csname PY@tok@gh\endcsname{\let\PY@bf=\textbf\def\PY@tc##1{\textcolor[rgb]{0.00,0.00,0.50}{##1}}}
\expandafter\def\csname PY@tok@gu\endcsname{\let\PY@bf=\textbf\def\PY@tc##1{\textcolor[rgb]{0.50,0.00,0.50}{##1}}}
\expandafter\def\csname PY@tok@gd\endcsname{\def\PY@tc##1{\textcolor[rgb]{0.63,0.00,0.00}{##1}}}
\expandafter\def\csname PY@tok@gi\endcsname{\def\PY@tc##1{\textcolor[rgb]{0.00,0.63,0.00}{##1}}}
\expandafter\def\csname PY@tok@gr\endcsname{\def\PY@tc##1{\textcolor[rgb]{1.00,0.00,0.00}{##1}}}
\expandafter\def\csname PY@tok@ge\endcsname{\let\PY@it=\textit}
\expandafter\def\csname PY@tok@gs\endcsname{\let\PY@bf=\textbf}
\expandafter\def\csname PY@tok@gp\endcsname{\let\PY@bf=\textbf\def\PY@tc##1{\textcolor[rgb]{0.00,0.00,0.50}{##1}}}
\expandafter\def\csname PY@tok@go\endcsname{\def\PY@tc##1{\textcolor[rgb]{0.53,0.53,0.53}{##1}}}
\expandafter\def\csname PY@tok@gt\endcsname{\def\PY@tc##1{\textcolor[rgb]{0.00,0.27,0.87}{##1}}}
\expandafter\def\csname PY@tok@err\endcsname{\def\PY@bc##1{\setlength{\fboxsep}{0pt}\fcolorbox[rgb]{1.00,0.00,0.00}{1,1,1}{\strut ##1}}}
\expandafter\def\csname PY@tok@kc\endcsname{\let\PY@bf=\textbf\def\PY@tc##1{\textcolor[rgb]{0.00,0.50,0.00}{##1}}}
\expandafter\def\csname PY@tok@kd\endcsname{\let\PY@bf=\textbf\def\PY@tc##1{\textcolor[rgb]{0.00,0.50,0.00}{##1}}}
\expandafter\def\csname PY@tok@kn\endcsname{\let\PY@bf=\textbf\def\PY@tc##1{\textcolor[rgb]{0.00,0.50,0.00}{##1}}}
\expandafter\def\csname PY@tok@kr\endcsname{\let\PY@bf=\textbf\def\PY@tc##1{\textcolor[rgb]{0.00,0.50,0.00}{##1}}}
\expandafter\def\csname PY@tok@bp\endcsname{\def\PY@tc##1{\textcolor[rgb]{0.00,0.50,0.00}{##1}}}
\expandafter\def\csname PY@tok@fm\endcsname{\def\PY@tc##1{\textcolor[rgb]{0.00,0.00,1.00}{##1}}}
\expandafter\def\csname PY@tok@vc\endcsname{\def\PY@tc##1{\textcolor[rgb]{0.10,0.09,0.49}{##1}}}
\expandafter\def\csname PY@tok@vg\endcsname{\def\PY@tc##1{\textcolor[rgb]{0.10,0.09,0.49}{##1}}}
\expandafter\def\csname PY@tok@vi\endcsname{\def\PY@tc##1{\textcolor[rgb]{0.10,0.09,0.49}{##1}}}
\expandafter\def\csname PY@tok@vm\endcsname{\def\PY@tc##1{\textcolor[rgb]{0.10,0.09,0.49}{##1}}}
\expandafter\def\csname PY@tok@sa\endcsname{\def\PY@tc##1{\textcolor[rgb]{0.73,0.13,0.13}{##1}}}
\expandafter\def\csname PY@tok@sb\endcsname{\def\PY@tc##1{\textcolor[rgb]{0.73,0.13,0.13}{##1}}}
\expandafter\def\csname PY@tok@sc\endcsname{\def\PY@tc##1{\textcolor[rgb]{0.73,0.13,0.13}{##1}}}
\expandafter\def\csname PY@tok@dl\endcsname{\def\PY@tc##1{\textcolor[rgb]{0.73,0.13,0.13}{##1}}}
\expandafter\def\csname PY@tok@s2\endcsname{\def\PY@tc##1{\textcolor[rgb]{0.73,0.13,0.13}{##1}}}
\expandafter\def\csname PY@tok@sh\endcsname{\def\PY@tc##1{\textcolor[rgb]{0.73,0.13,0.13}{##1}}}
\expandafter\def\csname PY@tok@s1\endcsname{\def\PY@tc##1{\textcolor[rgb]{0.73,0.13,0.13}{##1}}}
\expandafter\def\csname PY@tok@mb\endcsname{\def\PY@tc##1{\textcolor[rgb]{0.40,0.40,0.40}{##1}}}
\expandafter\def\csname PY@tok@mf\endcsname{\def\PY@tc##1{\textcolor[rgb]{0.40,0.40,0.40}{##1}}}
\expandafter\def\csname PY@tok@mh\endcsname{\def\PY@tc##1{\textcolor[rgb]{0.40,0.40,0.40}{##1}}}
\expandafter\def\csname PY@tok@mi\endcsname{\def\PY@tc##1{\textcolor[rgb]{0.40,0.40,0.40}{##1}}}
\expandafter\def\csname PY@tok@il\endcsname{\def\PY@tc##1{\textcolor[rgb]{0.40,0.40,0.40}{##1}}}
\expandafter\def\csname PY@tok@mo\endcsname{\def\PY@tc##1{\textcolor[rgb]{0.40,0.40,0.40}{##1}}}
\expandafter\def\csname PY@tok@ch\endcsname{\let\PY@it=\textit\def\PY@tc##1{\textcolor[rgb]{0.25,0.50,0.50}{##1}}}
\expandafter\def\csname PY@tok@cm\endcsname{\let\PY@it=\textit\def\PY@tc##1{\textcolor[rgb]{0.25,0.50,0.50}{##1}}}
\expandafter\def\csname PY@tok@cpf\endcsname{\let\PY@it=\textit\def\PY@tc##1{\textcolor[rgb]{0.25,0.50,0.50}{##1}}}
\expandafter\def\csname PY@tok@c1\endcsname{\let\PY@it=\textit\def\PY@tc##1{\textcolor[rgb]{0.25,0.50,0.50}{##1}}}
\expandafter\def\csname PY@tok@cs\endcsname{\let\PY@it=\textit\def\PY@tc##1{\textcolor[rgb]{0.25,0.50,0.50}{##1}}}

\def\PYZbs{\char`\\}
\def\PYZus{\char`\_}
\def\PYZob{\char`\{}
\def\PYZcb{\char`\}}
\def\PYZca{\char`\^}
\def\PYZam{\char`\&}
\def\PYZlt{\char`\<}
\def\PYZgt{\char`\>}
\def\PYZsh{\char`\#}
\def\PYZpc{\char`\%}
\def\PYZdl{\char`\$}
\def\PYZhy{\char`\-}
\def\PYZsq{\char`\'}
\def\PYZdq{\char`\"}
\def\PYZti{\char`\~}
% for compatibility with earlier versions
\def\PYZat{@}
\def\PYZlb{[}
\def\PYZrb{]}
\makeatother


    % Exact colors from NB
    \definecolor{incolor}{rgb}{0.0, 0.0, 0.5}
    \definecolor{outcolor}{rgb}{0.545, 0.0, 0.0}



    
    % Prevent overflowing lines due to hard-to-break entities
    \sloppy 
    % Setup hyperref package
    \hypersetup{
      breaklinks=true,  % so long urls are correctly broken across lines
      colorlinks=true,
      urlcolor=urlcolor,
      linkcolor=linkcolor,
      citecolor=citecolor,
      }
    % Slightly bigger margins than the latex defaults
    
    \geometry{verbose,tmargin=1in,bmargin=1in,lmargin=1in,rmargin=1in}
    
    

    \begin{document}
    
    
    \maketitle
    
    

    
    \#

MIS 382N: ADVANCED PREDICTIVE MODELING - MSBA

\#

Assignment 1

\#\#

Total points: 80

\#\#

Due: Tuesday, September 18 submitted via Canvas by 11:59 pm

Your homework should be written in a \textbf{Jupyter notebook}. You may
work in groups of two if you wish. Only one student per team needs to
submit the assignment on Canvas. But be sure to include name and UTEID
for both students. Homework groups will be created and managed through
Canvas, so please do not arbitrarily change your homework group. If you
do change, let the TA know.

Also, please make sure your code runs and the graphics (and anything
else) are displayed in your notebook before submitting. (\%matplotlib
inline)

    \section{Question 1: Challenges in Data Science (10
pts)}\label{question-1-challenges-in-data-science-10-pts}

Refer to the Domino guide under Modules -\/-\textgreater{} Additional
Resources

Section 2 describes 8 Challenges. You may have personally encountered or
heard of somebody else who encountered some of these challenge. If so,
please write 1-2 paragraphs on what situation was encountered and how it
mapped into one the mentioned challenges. If not, think of a
hypothetical case and do the same exercise.

\subsubsection{Try to solve the right problem but have the wrong
tools:}\label{try-to-solve-the-right-problem-but-have-the-wrong-tools}

Couple of years ago, I was working on patients charts data to understand
patient characteristics, prescribing behavior, physician perceptions and
brand loyalty for different drugs in the Multiple Myeloma market. I
wanted to run the random forest algorithm in R to better understand the
important patient characteristics associated with different drug types.
Our team had never worked on any advanced predictive models and hence it
took me at least a week to convince my manager to try using it. Even
after the approval from my manager the IT team refused to allow me the
permission to install the random forest package in R, citing data
security reasons. We eventually analyzed the data using traditional
tools such as Excel and in-house analytics platform, which limited my
analysis to the traditional insights. I believe concerns over data
security and slow acceptance of advanced analytical and statistical
tools such as R and Python by senior leadership is a major reasons for
having limited tools and resources to analyze data.

    \section{Question 2: Maximum likelihood estimate (10
pts)}\label{question-2-maximum-likelihood-estimate-10-pts}

Prove the statement on slide 3 of notes on MLR: "Then minimizing Mean
Squared Error (MSE) on the training data (which you can do using OLS)
yields the Maximum Likelihood Estimate (MLE) solution of the assumed
generative model."

\subsection{Answer}\label{answer}

    \begin{Verbatim}[commandchars=\\\{\}]
{\color{incolor}In [{\color{incolor}15}]:} \PY{o}{\PYZpc{}}\PY{k}{pylab} inline
         \PY{k+kn}{from} \PY{n+nn}{PIL} \PY{k}{import} \PY{n}{Image}
         \PY{n}{Image}\PY{o}{.}\PY{n}{open}\PY{p}{(}\PY{l+s+s2}{\PYZdq{}}\PY{l+s+s2}{Q2\PYZus{}1.jpg}\PY{l+s+s2}{\PYZdq{}}\PY{p}{)}
\end{Verbatim}


    \begin{Verbatim}[commandchars=\\\{\}]
Populating the interactive namespace from numpy and matplotlib

    \end{Verbatim}
\texttt{\color{outcolor}Out[{\color{outcolor}15}]:}
    
    \begin{center}
    \adjustimage{max size={0.9\linewidth}{0.9\paperheight}}{output_3_1.png}
    \end{center}
    { \hspace*{\fill} \\}
    

    \section{Question 3: Collinearity Issues (5
pts)}\label{question-3-collinearity-issues-5-pts}

What is the ``collinearity'' issue encountered in MLR? Why is this a
problem when you try to use MLR? Suggest one way of addressing this
problem.

    If the predictors (X's) in Multiple Linear Regression are correlated,
..............

Collinearity is a problem when Multiple Linear Regression because of the
following 2 reasons:

\begin{enumerate}
\def\labelenumi{\arabic{enumi}.}
\tightlist
\item
  Correlation between predictors affects the interpretability of
  parameters (Beta - coefficients)
\item
  Dependencies between predictors increases the stadard error of the
  parameters
\end{enumerate}

We can address this problem by either increasing the \# of data points,
which will reduce the standard error for the parameter (coefficients) or
by using new predictor variables which are less correlated.While adding
new variables could make the mean squared error to reduce, the standard
error of the parameters might increase.

    \section{Question 4: Multiple Linear Regression in Python (20
pts)}\label{question-4-multiple-linear-regression-in-python-20-pts}

Use the following code to import the allstate-claims-severity dataset
and linear models in python. The dataset is taken from
https://www.kaggle.com/c/allstate-claims-severity I have removed the
categorical variables to make it easier to run the models. Because of
this, MAE (Mean absolute error) will be on higher end compared to other
Kaggle entries. Please download the dataset "reduced\_train.csv" and use
it.

\subsection{Answer}\label{answer}

    \begin{Verbatim}[commandchars=\\\{\}]
{\color{incolor}In [{\color{incolor}16}]:} \PY{k+kn}{from} \PY{n+nn}{sklearn} \PY{k}{import} \PY{n}{linear\PYZus{}model}
         \PY{k+kn}{from} \PY{n+nn}{sklearn}\PY{n+nn}{.}\PY{n+nn}{metrics} \PY{k}{import} \PY{n}{mean\PYZus{}squared\PYZus{}error}\PY{p}{,} \PY{n}{mean\PYZus{}absolute\PYZus{}error}
         \PY{k+kn}{import} \PY{n+nn}{numpy} \PY{k}{as} \PY{n+nn}{np}
         \PY{k+kn}{import} \PY{n+nn}{pandas} \PY{k}{as} \PY{n+nn}{pd}
         \PY{k+kn}{import} \PY{n+nn}{matplotlib}\PY{n+nn}{.}\PY{n+nn}{pyplot} \PY{k}{as} \PY{n+nn}{plt}
         \PY{k+kn}{import} \PY{n+nn}{seaborn} \PY{k}{as} \PY{n+nn}{sns}
         
         \PY{n}{train\PYZus{}df} \PY{o}{=} \PY{n}{pd}\PY{o}{.}\PY{n}{read\PYZus{}csv}\PY{p}{(}\PY{l+s+s2}{\PYZdq{}}\PY{l+s+s2}{reduced\PYZus{}train.csv}\PY{l+s+s2}{\PYZdq{}}\PY{p}{)}
         
         \PY{n}{X} \PY{o}{=} \PY{n}{train\PYZus{}df}\PY{o}{.}\PY{n}{drop}\PY{p}{(}\PY{p}{[}\PY{l+s+s1}{\PYZsq{}}\PY{l+s+s1}{id}\PY{l+s+s1}{\PYZsq{}}\PY{p}{,}\PY{l+s+s1}{\PYZsq{}}\PY{l+s+s1}{loss}\PY{l+s+s1}{\PYZsq{}}\PY{p}{]}\PY{p}{,}\PY{n}{axis}\PY{o}{=}\PY{l+m+mi}{1}\PY{p}{)}
         \PY{n}{Y} \PY{o}{=} \PY{n}{train\PYZus{}df}\PY{p}{[}\PY{l+s+s1}{\PYZsq{}}\PY{l+s+s1}{loss}\PY{l+s+s1}{\PYZsq{}}\PY{p}{]}
\end{Verbatim}


    \begin{enumerate}
\def\labelenumi{\alph{enumi}.}
\tightlist
\item
  (2 pts) Print the shape (number of rows and columns) of the feature
  matrix X, and print the first 5 rows.
\end{enumerate}

    \begin{Verbatim}[commandchars=\\\{\}]
{\color{incolor}In [{\color{incolor}23}]:} \PY{c+c1}{\PYZsh{} Shape of X }
         \PY{n+nb}{print} \PY{p}{(}\PY{n}{X}\PY{o}{.}\PY{n}{shape}\PY{p}{)}
         \PY{c+c1}{\PYZsh{} Printing the first 5 rows}
         \PY{n}{X}\PY{o}{.}\PY{n}{head}\PY{p}{(}\PY{p}{)}
\end{Verbatim}


    \begin{Verbatim}[commandchars=\\\{\}]
(188318, 14)

    \end{Verbatim}

\begin{Verbatim}[commandchars=\\\{\}]
{\color{outcolor}Out[{\color{outcolor}23}]:}       cont1     cont2     cont3     cont4     cont5     cont6     cont7  \textbackslash{}
         0  0.726300  0.245921  0.187583  0.789639  0.310061  0.718367  0.335060   
         1  0.330514  0.737068  0.592681  0.614134  0.885834  0.438917  0.436585   
         2  0.261841  0.358319  0.484196  0.236924  0.397069  0.289648  0.315545   
         3  0.321594  0.555782  0.527991  0.373816  0.422268  0.440945  0.391128   
         4  0.273204  0.159990  0.527991  0.473202  0.704268  0.178193  0.247408   
         
              cont8    cont9   cont10    cont11    cont12    cont13    cont14  
         0  0.30260  0.67135  0.83510  0.569745  0.594646  0.822493  0.714843  
         1  0.60087  0.35127  0.43919  0.338312  0.366307  0.611431  0.304496  
         2  0.27320  0.26076  0.32446  0.381398  0.373424  0.195709  0.774425  
         3  0.31796  0.32128  0.44467  0.327915  0.321570  0.605077  0.602642  
         4  0.24564  0.22089  0.21230  0.204687  0.202213  0.246011  0.432606  
\end{Verbatim}
            
    \begin{enumerate}
\def\labelenumi{\alph{enumi}.}
\setcounter{enumi}{1}
\tightlist
\item
  (6 pts) Using ordinary least squares, fit a multiple linear regression
  (MLR) on all the feature variables using the entire dataset. Report
  the regression coefficient of each input feature and evaluate the
  model using mean absolute error (MAE). Example of ordinary least
  squares in Python is shown in Section 1.1.1 of
  http://scikit-learn.org/stable/modules/linear\_model.html.
\end{enumerate}

    \begin{Verbatim}[commandchars=\\\{\}]
{\color{incolor}In [{\color{incolor}42}]:} \PY{c+c1}{\PYZsh{} Fitting multiple linear regression on all the features using the entire data set using OLS}
         \PY{n}{reg} \PY{o}{=} \PY{n}{linear\PYZus{}model}\PY{o}{.}\PY{n}{LinearRegression}\PY{p}{(}\PY{p}{)}
         \PY{n}{reg}\PY{o}{.}\PY{n}{fit} \PY{p}{(}\PY{n}{X}\PY{p}{,} \PY{n}{Y}\PY{p}{)}
         
         \PY{c+c1}{\PYZsh{}Calculating regression coefficient for each input feature}
         
         \PY{n}{pd}\PY{o}{.}\PY{n}{concat}\PY{p}{(}\PY{p}{[}\PY{n}{pd}\PY{o}{.}\PY{n}{DataFrame}\PY{p}{(}\PY{n}{X}\PY{o}{.}\PY{n}{columns}\PY{o}{.}\PY{n}{values}\PY{p}{,}\PY{n}{columns} \PY{o}{=} \PY{p}{[}\PY{l+s+s1}{\PYZsq{}}\PY{l+s+s1}{Features}\PY{l+s+s1}{\PYZsq{}}\PY{p}{]}\PY{p}{)}\PY{p}{,}\PYZbs{}
                    \PY{n}{pd}\PY{o}{.}\PY{n}{DataFrame}\PY{p}{(}\PY{n}{reg}\PY{o}{.}\PY{n}{coef\PYZus{}} \PY{p}{,}\PY{n}{columns} \PY{o}{=} \PY{p}{[}\PY{l+s+s1}{\PYZsq{}}\PY{l+s+s1}{Regression\PYZus{}coefficients}\PY{l+s+s1}{\PYZsq{}}\PY{p}{]}\PY{p}{)}\PY{p}{]}\PY{p}{,}\PY{n}{axis} \PY{o}{=}\PY{l+m+mi}{1}\PY{p}{)}
\end{Verbatim}


\begin{Verbatim}[commandchars=\\\{\}]
{\color{outcolor}Out[{\color{outcolor}42}]:}    Features  Regression\_coefficients
         0     cont1             -2137.111212
         1     cont2              1762.240616
         2     cont3               -56.579685
         3     cont4              -301.172981
         4     cont5                30.032018
         5     cont6              -557.225099
         6     cont7              1707.955206
         7     cont8               298.612406
         8     cont9              2049.379104
         9    cont10              -250.019861
         10   cont11             -1732.348874
         11   cont12              2724.389858
         12   cont13              -631.301289
         13   cont14               273.626005
\end{Verbatim}
            
    \begin{Verbatim}[commandchars=\\\{\}]
{\color{incolor}In [{\color{incolor}44}]:} \PY{c+c1}{\PYZsh{}Calculating the Mean Absolute Error}
         \PY{n+nb}{print} \PY{p}{(}\PY{l+s+s1}{\PYZsq{}}\PY{l+s+s1}{Mean Absolute Error of the Multiple Linear regression Model is :}\PY{l+s+s1}{\PYZsq{}}\PY{p}{,}\PY{n}{mean\PYZus{}absolute\PYZus{}error}\PY{p}{(}\PY{n}{Y}\PY{p}{,}\PY{n}{reg}\PY{o}{.}\PY{n}{predict}\PY{p}{(}\PY{n}{X}\PY{p}{)}\PY{p}{)}\PY{p}{)}
\end{Verbatim}


    \begin{Verbatim}[commandchars=\\\{\}]
Mean Absolute Error of the Linear regression Model is : 1950.3606265639835

    \end{Verbatim}

    \begin{enumerate}
\def\labelenumi{\alph{enumi}.}
\setcounter{enumi}{2}
\tightlist
\item
  (6 pts) Split the data into a training set and a test set, using the
  train\_test\_split with test\_size = 0.30 and random\_state = 50. The
  code for this is given below. Fit an MLR using the training set.
  Evaluate the trained model using the training set and the test set,
  respectively. Compare the two MAE values thus obtained.
\end{enumerate}

    \begin{Verbatim}[commandchars=\\\{\}]
{\color{incolor}In [{\color{incolor}239}]:} \PY{k+kn}{from} \PY{n+nn}{sklearn}\PY{n+nn}{.}\PY{n+nn}{model\PYZus{}selection} \PY{k}{import} \PY{n}{train\PYZus{}test\PYZus{}split}
          
          \PY{n}{X\PYZus{}train}\PY{p}{,} \PY{n}{X\PYZus{}test}\PY{p}{,} \PY{n}{Y\PYZus{}train}\PY{p}{,} \PY{n}{Y\PYZus{}test} \PY{o}{=} \PY{n}{train\PYZus{}test\PYZus{}split}\PY{p}{(}\PY{n}{X}\PY{p}{,}\PY{n}{Y}\PY{p}{,}\PY{n}{test\PYZus{}size} \PY{o}{=} \PY{l+m+mf}{0.30}\PY{p}{,} \PY{n}{random\PYZus{}state}\PY{o}{=}\PY{l+m+mi}{50}\PY{p}{)}
          \PY{n}{reg} \PY{o}{=} \PY{n}{linear\PYZus{}model}\PY{o}{.}\PY{n}{LinearRegression}\PY{p}{(}\PY{p}{)}
          \PY{n}{reg}\PY{o}{.}\PY{n}{fit} \PY{p}{(}\PY{n}{X\PYZus{}train}\PY{p}{,} \PY{n}{Y\PYZus{}train}\PY{p}{)}
          \PY{c+c1}{\PYZsh{} Prediction on Training Data}
          \PY{n}{prediction\PYZus{}train} \PY{o}{=} \PY{n}{reg}\PY{o}{.}\PY{n}{predict}\PY{p}{(}\PY{n}{X\PYZus{}train}\PY{p}{)}
          \PY{n}{MAE\PYZus{}train} \PY{o}{=} \PY{n}{mean\PYZus{}absolute\PYZus{}error}\PY{p}{(}\PY{n}{Y\PYZus{}train}\PY{p}{,}\PY{n}{prediction\PYZus{}train}\PY{p}{)}
          
          \PY{c+c1}{\PYZsh{}Predicton on the Test Data }
          \PY{n}{prediction\PYZus{}test} \PY{o}{=} \PY{n}{reg}\PY{o}{.}\PY{n}{predict}\PY{p}{(}\PY{n}{X\PYZus{}test}\PY{p}{)}
          \PY{n}{MAE\PYZus{}test} \PY{o}{=} \PY{n}{mean\PYZus{}absolute\PYZus{}error}\PY{p}{(}\PY{n}{Y\PYZus{}test}\PY{p}{,}\PY{n}{prediction\PYZus{}test}\PY{p}{)}
          
          
          \PY{n+nb}{print} \PY{p}{(}\PY{l+s+s1}{\PYZsq{}}\PY{l+s+s1}{Mean Absolute Error from the trained data is :}\PY{l+s+s1}{\PYZsq{}}\PY{p}{,} \PY{n}{MAE\PYZus{}train}\PY{p}{)}
          \PY{n+nb}{print} \PY{p}{(}\PY{l+s+s1}{\PYZsq{}}\PY{l+s+s1}{Mean Absolute Error from the test data is :}\PY{l+s+s1}{\PYZsq{}}\PY{p}{,} \PY{n}{MAE\PYZus{}test}\PY{p}{)}
\end{Verbatim}


    \begin{Verbatim}[commandchars=\\\{\}]
Mean Absolute Error from the trained data is : 1946.9866773014621
Mean Absolute Error from the test data is : 1954.111059596165

    \end{Verbatim}

    The MAE obtained on the test data is slightly higher

    \begin{enumerate}
\def\labelenumi{\alph{enumi}.}
\setcounter{enumi}{3}
\tightlist
\item
  (6 pts) Plot the histogram of Y\_train and see its distribution. Now
  take log of Y\_train and plot its histogram. Now run regression again
  after taking log and compare the MAE. You need to do
  np.exp(predictions) to bring them back to original scale, and then
  calculate MAE. Explain the results.
\end{enumerate}

    \begin{Verbatim}[commandchars=\\\{\}]
{\color{incolor}In [{\color{incolor}198}]:} \PY{c+c1}{\PYZsh{} Plotting histograms for Y\PYZus{}train and log(Y\PYZus{}train)}
          \PY{n}{fig}\PY{p}{,} \PY{p}{(}\PY{n}{ax1}\PY{p}{,}\PY{n}{ax2}\PY{p}{)} \PY{o}{=} \PY{n}{plt}\PY{o}{.}\PY{n}{subplots}\PY{p}{(}\PY{n}{nrows}\PY{o}{=}\PY{l+m+mi}{1}\PY{p}{,} \PY{n}{ncols}\PY{o}{=}\PY{l+m+mi}{2}\PY{p}{,} \PY{n}{sharey} \PY{o}{=} \PY{k+kc}{False}\PY{p}{,} \PY{n}{figsize} \PY{o}{=} \PY{p}{(}\PY{l+m+mi}{15}\PY{p}{,}\PY{l+m+mi}{5}\PY{p}{)}\PY{p}{)}
          
          \PY{n}{sns}\PY{o}{.}\PY{n}{distplot}\PY{p}{(}\PY{n}{Y\PYZus{}train}\PY{p}{,}\PY{n}{hist} \PY{o}{=} \PY{k+kc}{True}\PY{p}{,} \PY{n}{ax} \PY{o}{=} \PY{n}{ax1}\PY{p}{)}
          \PY{n}{sns}\PY{o}{.}\PY{n}{distplot}\PY{p}{(}\PY{n}{log}\PY{p}{(}\PY{n}{Y\PYZus{}train}\PY{p}{)}\PY{p}{,}\PY{n}{hist} \PY{o}{=} \PY{k+kc}{True}\PY{p}{,} \PY{n}{ax} \PY{o}{=} \PY{n}{ax2}\PY{p}{)}
\end{Verbatim}


    \begin{Verbatim}[commandchars=\\\{\}]
C:\textbackslash{}Users\textbackslash{}bandi\textbackslash{}Anaconda3\textbackslash{}lib\textbackslash{}site-packages\textbackslash{}matplotlib\textbackslash{}axes\textbackslash{}\_axes.py:6462: UserWarning: The 'normed' kwarg is deprecated, and has been replaced by the 'density' kwarg.
  warnings.warn("The 'normed' kwarg is deprecated, and has been "
C:\textbackslash{}Users\textbackslash{}bandi\textbackslash{}Anaconda3\textbackslash{}lib\textbackslash{}site-packages\textbackslash{}matplotlib\textbackslash{}axes\textbackslash{}\_axes.py:6462: UserWarning: The 'normed' kwarg is deprecated, and has been replaced by the 'density' kwarg.
  warnings.warn("The 'normed' kwarg is deprecated, and has been "

    \end{Verbatim}

\begin{Verbatim}[commandchars=\\\{\}]
{\color{outcolor}Out[{\color{outcolor}198}]:} <matplotlib.axes.\_subplots.AxesSubplot at 0x1df86894b38>
\end{Verbatim}
            
    \begin{center}
    \adjustimage{max size={0.9\linewidth}{0.9\paperheight}}{output_17_2.png}
    \end{center}
    { \hspace*{\fill} \\}
    
    \begin{Verbatim}[commandchars=\\\{\}]
{\color{incolor}In [{\color{incolor}240}]:} \PY{n}{log\PYZus{}Y\PYZus{}train} \PY{o}{=} \PY{n}{log}\PY{p}{(}\PY{n}{Y\PYZus{}train}\PY{p}{)}
          \PY{n}{reg}\PY{o}{.}\PY{n}{fit} \PY{p}{(}\PY{n}{X\PYZus{}train}\PY{p}{,} \PY{n}{log\PYZus{}Y\PYZus{}train}\PY{p}{)}
          \PY{n}{prediction\PYZus{}test} \PY{o}{=} \PY{n}{np}\PY{o}{.}\PY{n}{exp}\PY{p}{(}\PY{n}{reg}\PY{o}{.}\PY{n}{predict}\PY{p}{(}\PY{n}{X\PYZus{}test}\PY{p}{)}\PY{p}{)}
          \PY{n}{MAE\PYZus{}test} \PY{o}{=} \PY{n}{mean\PYZus{}absolute\PYZus{}error}\PY{p}{(}\PY{n}{Y\PYZus{}test}\PY{p}{,}\PY{n}{prediction\PYZus{}test}\PY{p}{)}
          
          \PY{c+c1}{\PYZsh{}prediction\PYZus{}test}
          
          \PY{n+nb}{print} \PY{p}{(}\PY{l+s+s1}{\PYZsq{}}\PY{l+s+s1}{The new MAE is :}\PY{l+s+s1}{\PYZsq{}}\PY{p}{,}\PY{n}{MAE\PYZus{}test}\PY{p}{)}
\end{Verbatim}


    \begin{Verbatim}[commandchars=\\\{\}]
The new MAE is : 1810.6519682594767

    \end{Verbatim}

    The MAE reduced after taking log (Y\_train), because Y\_train was not
normally distributed. Hence it was violating the linear regression's
assumption. Once we took log (Y\_train), the predictor became normally
distributed, thus reducing the MAE.

    \section{Question 5: Ridge and Lasso Regression (25
points)}\label{question-5-ridge-and-lasso-regression-25-points}

    Using the same data from before, in this question you will explore the
application of Lasso and Ridge regression using sklearn package in
Python. Split the data into a training set and a test set, using the
train\_test\_split with test\_size = 0.30 and random\_state = 50. Take
log of the Y\_train and use it for training.

    \begin{enumerate}
\def\labelenumi{\arabic{enumi})}
\item
  Use sklearn.linear\_model.Lasso and sklearn.linear\_model.Ridge
  classes to do a
  \href{http://scikit-learn.org/stable/auto_examples/exercises/plot_cv_diabetes.html\#example-exercises-plot-cv-diabetes-py}{5-fold
  cross validation} using sklearn's
  \href{http://scikit-learn.org/stable/modules/generated/sklearn.cross_validation.KFold.html}{KFold}.
  For the sweep of the regularization parameter, we will look at a grid
  of values ranging from \(\lambda = 10^{10}\) to \(\lambda = 10^{-6}\).
  In Python, you can consider this range of values as follows:

  import numpy as np

  alphas = 10**np.linspace(10,-6,100)*0.5
\end{enumerate}

Report the best chosen \(\lambda\) based on cross validation. The cross
validation should happen on your training data using average MAE as the
scoring metric. (8pts)

\begin{enumerate}
\def\labelenumi{\arabic{enumi})}
\setcounter{enumi}{1}
\item
  Run ridge and lasso for all of the alphas specified above (on training
  data), and plot the coefficients learned for each of them - there
  should be one plot each for lasso and ridge, so a total of two plots;
  the plots for different features for a method should be on the same
  plot. What do you qualitatively observe when value of the
  regularization parameter is changed? (7pts)
\item
  Run least squares regression, ridge, and lasso on the training data.
  For ridge and lasso, use only the best regularization parameter.
  Report the prediction error (MAE) on the test data for each. (5pts)
\item
  Run lasso again with cross validation using
  \href{http://scikit-learn.org/stable/modules/generated/sklearn.linear_model.LassoCV.html}{sklearn.linear\_model.LassoCV}.
  Set the cross validation parameters as follows:

  LassoCV(alphas=None, cv=10, max\_iter=10000)
\end{enumerate}

Report the best \(\lambda\) based on cross validation. Run lasso on the
training data using the best \(\lambda\) and report the coefficeints for
all variables. (5pts)

\subsection{Answer}\label{answer}

    \begin{Verbatim}[commandchars=\\\{\}]
{\color{incolor}In [{\color{incolor}235}]:} \PY{c+c1}{\PYZsh{} Lasso Regression}
          \PY{k+kn}{from} \PY{n+nn}{sklearn}\PY{n+nn}{.}\PY{n+nn}{linear\PYZus{}model} \PY{k}{import} \PY{n}{Lasso}
          \PY{k+kn}{from} \PY{n+nn}{sklearn}\PY{n+nn}{.}\PY{n+nn}{model\PYZus{}selection} \PY{k}{import} \PY{n}{KFold}
          
          \PY{c+c1}{\PYZsh{} Setting the range for alpha and \PYZsh{} of k folds required}
          \PY{n}{alphas} \PY{o}{=}  \PY{l+m+mi}{10}\PY{o}{*}\PY{o}{*}\PY{n}{np}\PY{o}{.}\PY{n}{linspace}\PY{p}{(}\PY{l+m+mi}{10}\PY{p}{,}\PY{o}{\PYZhy{}}\PY{l+m+mi}{6}\PY{p}{,}\PY{l+m+mi}{100}\PY{p}{)}\PY{o}{*}\PY{l+m+mf}{0.5}
          \PY{n}{kf} \PY{o}{=} \PY{n}{KFold}\PY{p}{(}\PY{n}{n\PYZus{}splits}\PY{o}{=}\PY{l+m+mi}{5}\PY{p}{)}
          \PY{n}{MAE\PYZus{}cv} \PY{o}{=} \PY{n}{pd}\PY{o}{.}\PY{n}{DataFrame}\PY{p}{(}\PY{n}{columns} \PY{o}{=} \PY{n}{alphas}\PY{p}{)}
          
          \PY{c+c1}{\PYZsh{} Loop for 5 cross validations}
          \PY{k}{for} \PY{n}{train\PYZus{}index}\PY{p}{,} \PY{n}{test\PYZus{}index} \PY{o+ow}{in} \PY{n}{kf}\PY{o}{.}\PY{n}{split}\PY{p}{(}\PY{n}{X\PYZus{}train}\PY{p}{)}\PY{p}{:}
              \PY{n}{X\PYZus{}train\PYZus{}cv}\PY{p}{,} \PY{n}{X\PYZus{}test\PYZus{}cv} \PY{o}{=} \PY{n}{np}\PY{o}{.}\PY{n}{array}\PY{p}{(}\PY{n}{X\PYZus{}train}\PY{p}{)}\PY{p}{[}\PY{n}{train\PYZus{}index}\PY{p}{]}\PY{p}{,} \PY{n}{np}\PY{o}{.}\PY{n}{array}\PY{p}{(}\PY{n}{X\PYZus{}train}\PY{p}{)}\PY{p}{[}\PY{n}{test\PYZus{}index}\PY{p}{]}
              \PY{n}{y\PYZus{}train}\PY{p}{,} \PY{n}{y\PYZus{}test} \PY{o}{=} \PY{n}{np}\PY{o}{.}\PY{n}{array}\PY{p}{(}\PY{n}{log\PYZus{}Y\PYZus{}train}\PY{p}{)}\PY{p}{[}\PY{n}{train\PYZus{}index}\PY{p}{]}\PY{p}{,} \PY{n}{np}\PY{o}{.}\PY{n}{array}\PY{p}{(}\PY{n}{log\PYZus{}Y\PYZus{}train}\PY{p}{)}\PY{p}{[}\PY{n}{test\PYZus{}index}\PY{p}{]}
              \PY{n}{MAE} \PY{o}{=} \PY{p}{[}\PY{p}{]}
              \PY{c+c1}{\PYZsh{} Loop for all the alphas (lambda)}
              \PY{k}{for} \PY{n}{i} \PY{o+ow}{in} \PY{n+nb}{range}\PY{p}{(}\PY{n+nb}{len}\PY{p}{(}\PY{n}{alphas}\PY{p}{)}\PY{p}{)}\PY{p}{:}
                  \PY{n}{reg} \PY{o}{=} \PY{n}{linear\PYZus{}model}\PY{o}{.}\PY{n}{Lasso}\PY{p}{(}\PY{n}{alpha} \PY{o}{=} \PY{n}{alphas}\PY{p}{[}\PY{n}{i}\PY{p}{]}\PY{p}{)}
                  \PY{n}{reg}\PY{o}{.}\PY{n}{fit}\PY{p}{(}\PY{n}{X\PYZus{}train\PYZus{}cv}\PY{p}{,}\PY{n}{y\PYZus{}train}\PY{p}{)}
                  \PY{n}{prediction} \PY{o}{=} \PY{n}{np}\PY{o}{.}\PY{n}{exp}\PY{p}{(}\PY{n}{reg}\PY{o}{.}\PY{n}{predict}\PY{p}{(}\PY{n}{X\PYZus{}test\PYZus{}cv}\PY{p}{)}\PY{p}{)}
                  \PY{n}{MAE}\PY{o}{.}\PY{n}{append}\PY{p}{(}\PY{n}{mean\PYZus{}absolute\PYZus{}error}\PY{p}{(}\PY{n}{y\PYZus{}test}\PY{p}{,}\PY{n}{prediction}\PY{p}{)}\PY{p}{)}
              \PY{n}{MAE\PYZus{}cv}\PY{o}{.}\PY{n}{loc}\PY{p}{[}\PY{n+nb}{len}\PY{p}{(}\PY{n}{MAE\PYZus{}cv}\PY{p}{)}\PY{p}{]} \PY{o}{=} \PY{n}{MAE}
              
          \PY{c+c1}{\PYZsh{} Calculating the minimum lambda based on the average MAE }
          \PY{n}{lambda\PYZus{}min\PYZus{}lasso} \PY{o}{=} \PY{n}{MAE\PYZus{}cv}\PY{o}{.}\PY{n}{mean}\PY{p}{(}\PY{n}{axis} \PY{o}{=}\PY{l+m+mi}{0}\PY{p}{)}\PY{o}{.}\PY{n}{idxmin}\PY{p}{(}\PY{p}{)}
          
          \PY{c+c1}{\PYZsh{} Plotting Avergae MAE against lambda}
          \PY{n}{plt}\PY{o}{.}\PY{n}{semilogx}\PY{p}{(}\PY{n}{alphas}\PY{p}{,}\PY{n}{MAE\PYZus{}cv}\PY{o}{.}\PY{n}{mean}\PY{p}{(}\PY{n}{axis} \PY{o}{=}\PY{l+m+mi}{0}\PY{p}{)}\PY{o}{.}\PY{n}{values}\PY{p}{)}
          \PY{n}{plt}\PY{o}{.}\PY{n}{xlabel}\PY{p}{(}\PY{l+s+s1}{\PYZsq{}}\PY{l+s+s1}{Lambda}\PY{l+s+s1}{\PYZsq{}}\PY{p}{)}
          \PY{n}{plt}\PY{o}{.}\PY{n}{ylabel}\PY{p}{(}\PY{l+s+s1}{\PYZsq{}}\PY{l+s+s1}{Average Mean Absolute Error}\PY{l+s+s1}{\PYZsq{}}\PY{p}{)}
          
          \PY{n+nb}{print} \PY{p}{(}\PY{l+s+s1}{\PYZsq{}}\PY{l+s+s1}{Minimum lambda based on average MAE from Lasso Regression is :}\PY{l+s+s1}{\PYZsq{}}\PY{p}{,} \PY{n}{lambda\PYZus{}min\PYZus{}lasso}\PY{p}{)}
\end{Verbatim}


    \begin{Verbatim}[commandchars=\\\{\}]
C:\textbackslash{}Users\textbackslash{}bandi\textbackslash{}Anaconda3\textbackslash{}lib\textbackslash{}site-packages\textbackslash{}sklearn\textbackslash{}linear\_model\textbackslash{}coordinate\_descent.py:491: ConvergenceWarning: Objective did not converge. You might want to increase the number of iterations. Fitting data with very small alpha may cause precision problems.
  ConvergenceWarning)
C:\textbackslash{}Users\textbackslash{}bandi\textbackslash{}Anaconda3\textbackslash{}lib\textbackslash{}site-packages\textbackslash{}sklearn\textbackslash{}linear\_model\textbackslash{}coordinate\_descent.py:491: ConvergenceWarning: Objective did not converge. You might want to increase the number of iterations. Fitting data with very small alpha may cause precision problems.
  ConvergenceWarning)
C:\textbackslash{}Users\textbackslash{}bandi\textbackslash{}Anaconda3\textbackslash{}lib\textbackslash{}site-packages\textbackslash{}sklearn\textbackslash{}linear\_model\textbackslash{}coordinate\_descent.py:491: ConvergenceWarning: Objective did not converge. You might want to increase the number of iterations. Fitting data with very small alpha may cause precision problems.
  ConvergenceWarning)
C:\textbackslash{}Users\textbackslash{}bandi\textbackslash{}Anaconda3\textbackslash{}lib\textbackslash{}site-packages\textbackslash{}sklearn\textbackslash{}linear\_model\textbackslash{}coordinate\_descent.py:491: ConvergenceWarning: Objective did not converge. You might want to increase the number of iterations. Fitting data with very small alpha may cause precision problems.
  ConvergenceWarning)
C:\textbackslash{}Users\textbackslash{}bandi\textbackslash{}Anaconda3\textbackslash{}lib\textbackslash{}site-packages\textbackslash{}sklearn\textbackslash{}linear\_model\textbackslash{}coordinate\_descent.py:491: ConvergenceWarning: Objective did not converge. You might want to increase the number of iterations. Fitting data with very small alpha may cause precision problems.
  ConvergenceWarning)

    \end{Verbatim}

    \begin{Verbatim}[commandchars=\\\{\}]
Minimum lambda based on average MAE from Lasso Regression is : 0.01675801325469417

    \end{Verbatim}

    \begin{center}
    \adjustimage{max size={0.9\linewidth}{0.9\paperheight}}{output_23_2.png}
    \end{center}
    { \hspace*{\fill} \\}
    
    \begin{Verbatim}[commandchars=\\\{\}]
{\color{incolor}In [{\color{incolor}236}]:} \PY{c+c1}{\PYZsh{} Ridge Regression \PYZhy{} We will repeat the entire process for Ridge Regression }
          \PY{k+kn}{from} \PY{n+nn}{sklearn}\PY{n+nn}{.}\PY{n+nn}{linear\PYZus{}model} \PY{k}{import} \PY{n}{Ridge}
          \PY{k+kn}{from} \PY{n+nn}{sklearn}\PY{n+nn}{.}\PY{n+nn}{model\PYZus{}selection} \PY{k}{import} \PY{n}{KFold}
          
          \PY{c+c1}{\PYZsh{} Setting the rang for alpha and \PYZsh{} of k folds required}
          \PY{n}{alphas} \PY{o}{=}  \PY{l+m+mi}{10}\PY{o}{*}\PY{o}{*}\PY{n}{np}\PY{o}{.}\PY{n}{linspace}\PY{p}{(}\PY{l+m+mi}{10}\PY{p}{,}\PY{o}{\PYZhy{}}\PY{l+m+mi}{6}\PY{p}{,}\PY{l+m+mi}{100}\PY{p}{)}\PY{o}{*}\PY{l+m+mf}{0.5}
          \PY{n}{kf} \PY{o}{=} \PY{n}{KFold}\PY{p}{(}\PY{n}{n\PYZus{}splits}\PY{o}{=}\PY{l+m+mi}{5}\PY{p}{)}
          \PY{n}{MAE\PYZus{}cv} \PY{o}{=} \PY{n}{pd}\PY{o}{.}\PY{n}{DataFrame}\PY{p}{(}\PY{n}{columns} \PY{o}{=} \PY{n}{alphas}\PY{p}{)}
          
          \PY{c+c1}{\PYZsh{} Loop for 5 cross validations}
          \PY{k}{for} \PY{n}{train\PYZus{}index}\PY{p}{,} \PY{n}{test\PYZus{}index} \PY{o+ow}{in} \PY{n}{kf}\PY{o}{.}\PY{n}{split}\PY{p}{(}\PY{n}{X\PYZus{}train}\PY{p}{)}\PY{p}{:}
              \PY{n}{X\PYZus{}train\PYZus{}cv}\PY{p}{,} \PY{n}{X\PYZus{}test\PYZus{}cv} \PY{o}{=} \PY{n}{np}\PY{o}{.}\PY{n}{array}\PY{p}{(}\PY{n}{X\PYZus{}train}\PY{p}{)}\PY{p}{[}\PY{n}{train\PYZus{}index}\PY{p}{]}\PY{p}{,} \PY{n}{np}\PY{o}{.}\PY{n}{array}\PY{p}{(}\PY{n}{X\PYZus{}train}\PY{p}{)}\PY{p}{[}\PY{n}{test\PYZus{}index}\PY{p}{]}
              \PY{n}{y\PYZus{}train}\PY{p}{,} \PY{n}{y\PYZus{}test} \PY{o}{=} \PY{n}{np}\PY{o}{.}\PY{n}{array}\PY{p}{(}\PY{n}{log\PYZus{}Y\PYZus{}train}\PY{p}{)}\PY{p}{[}\PY{n}{train\PYZus{}index}\PY{p}{]}\PY{p}{,} \PY{n}{np}\PY{o}{.}\PY{n}{array}\PY{p}{(}\PY{n}{log\PYZus{}Y\PYZus{}train}\PY{p}{)}\PY{p}{[}\PY{n}{test\PYZus{}index}\PY{p}{]}
              \PY{n}{MAE} \PY{o}{=} \PY{p}{[}\PY{p}{]}
              \PY{c+c1}{\PYZsh{} Loop for all the alphas (lambda)}
              \PY{k}{for} \PY{n}{i} \PY{o+ow}{in} \PY{n+nb}{range}\PY{p}{(}\PY{n+nb}{len}\PY{p}{(}\PY{n}{alphas}\PY{p}{)}\PY{p}{)}\PY{p}{:}
                  \PY{n}{reg} \PY{o}{=} \PY{n}{linear\PYZus{}model}\PY{o}{.}\PY{n}{Ridge}\PY{p}{(}\PY{n}{alpha} \PY{o}{=} \PY{n}{alphas}\PY{p}{[}\PY{n}{i}\PY{p}{]}\PY{p}{)}
                  \PY{n}{reg}\PY{o}{.}\PY{n}{fit}\PY{p}{(}\PY{n}{X\PYZus{}train\PYZus{}cv}\PY{p}{,}\PY{n}{y\PYZus{}train}\PY{p}{)}
                  \PY{n}{prediction} \PY{o}{=} \PY{n}{np}\PY{o}{.}\PY{n}{exp}\PY{p}{(}\PY{n}{reg}\PY{o}{.}\PY{n}{predict}\PY{p}{(}\PY{n}{X\PYZus{}test\PYZus{}cv}\PY{p}{)}\PY{p}{)}
                  \PY{n}{MAE}\PY{o}{.}\PY{n}{append}\PY{p}{(}\PY{n}{mean\PYZus{}absolute\PYZus{}error}\PY{p}{(}\PY{n}{y\PYZus{}test}\PY{p}{,}\PY{n}{prediction}\PY{p}{)}\PY{p}{)}
              \PY{n}{MAE\PYZus{}cv}\PY{o}{.}\PY{n}{loc}\PY{p}{[}\PY{n+nb}{len}\PY{p}{(}\PY{n}{MAE\PYZus{}cv}\PY{p}{)}\PY{p}{]} \PY{o}{=} \PY{n}{MAE}
              
          \PY{c+c1}{\PYZsh{} Calculating the minimum lambda based on the average MAE }
          \PY{n}{lambda\PYZus{}min\PYZus{}ridge} \PY{o}{=} \PY{n}{MAE\PYZus{}cv}\PY{o}{.}\PY{n}{mean}\PY{p}{(}\PY{n}{axis} \PY{o}{=}\PY{l+m+mi}{0}\PY{p}{)}\PY{o}{.}\PY{n}{idxmin}\PY{p}{(}\PY{p}{)}
          
          \PY{c+c1}{\PYZsh{} Plotting Avergae MAE against lambda}
          \PY{n}{plt}\PY{o}{.}\PY{n}{semilogx}\PY{p}{(}\PY{n}{alphas}\PY{p}{,}\PY{n}{MAE\PYZus{}cv}\PY{o}{.}\PY{n}{mean}\PY{p}{(}\PY{n}{axis} \PY{o}{=}\PY{l+m+mi}{0}\PY{p}{)}\PY{o}{.}\PY{n}{values}\PY{p}{)}
          \PY{n}{plt}\PY{o}{.}\PY{n}{xlabel}\PY{p}{(}\PY{l+s+s1}{\PYZsq{}}\PY{l+s+s1}{Lambda}\PY{l+s+s1}{\PYZsq{}}\PY{p}{)}
          \PY{n}{plt}\PY{o}{.}\PY{n}{ylabel}\PY{p}{(}\PY{l+s+s1}{\PYZsq{}}\PY{l+s+s1}{Average Mean Absolute Error}\PY{l+s+s1}{\PYZsq{}}\PY{p}{)}
          
          \PY{n+nb}{print} \PY{p}{(}\PY{l+s+s1}{\PYZsq{}}\PY{l+s+s1}{Minimum lambda based on average MAE from Ridge Regression is :}\PY{l+s+s1}{\PYZsq{}}\PY{p}{,} \PY{n}{lambda\PYZus{}min\PYZus{}ridge}\PY{p}{)}
\end{Verbatim}


    \begin{Verbatim}[commandchars=\\\{\}]
Minimum lambda based on average MAE from Ridge Regression is : 27311386.088421684

    \end{Verbatim}

    \begin{center}
    \adjustimage{max size={0.9\linewidth}{0.9\paperheight}}{output_24_1.png}
    \end{center}
    { \hspace*{\fill} \\}
    
    \begin{Verbatim}[commandchars=\\\{\}]
{\color{incolor}In [{\color{incolor}229}]:} \PY{c+c1}{\PYZsh{} Running Lasso Regression for all the alpha values}
          \PY{n}{alphas} \PY{o}{=}  \PY{l+m+mi}{10}\PY{o}{*}\PY{o}{*}\PY{n}{np}\PY{o}{.}\PY{n}{linspace}\PY{p}{(}\PY{l+m+mi}{10}\PY{p}{,}\PY{o}{\PYZhy{}}\PY{l+m+mi}{6}\PY{p}{,}\PY{l+m+mi}{100}\PY{p}{)}\PY{o}{*}\PY{l+m+mf}{0.5}
          \PY{n}{coefs} \PY{o}{=} \PY{p}{[}\PY{p}{]}
          \PY{k}{for} \PY{n}{i} \PY{o+ow}{in} \PY{n+nb}{range}\PY{p}{(}\PY{n+nb}{len}\PY{p}{(}\PY{n}{alphas}\PY{p}{)}\PY{p}{)}\PY{p}{:}
                  \PY{n}{reg\PYZus{}lasso} \PY{o}{=} \PY{n}{linear\PYZus{}model}\PY{o}{.}\PY{n}{Lasso}\PY{p}{(}\PY{n}{alpha} \PY{o}{=} \PY{n}{alphas}\PY{p}{[}\PY{n}{i}\PY{p}{]}\PY{p}{)}
                  \PY{n}{reg\PYZus{}lasso}\PY{o}{.}\PY{n}{fit}\PY{p}{(}\PY{n}{X\PYZus{}train}\PY{p}{,}\PY{n}{log\PYZus{}Y\PYZus{}train}\PY{p}{)}
                  \PY{n}{prediction} \PY{o}{=} \PY{n}{np}\PY{o}{.}\PY{n}{exp}\PY{p}{(}\PY{n}{reg\PYZus{}lasso}\PY{o}{.}\PY{n}{predict}\PY{p}{(}\PY{n}{X\PYZus{}test}\PY{p}{)}\PY{p}{)}
                  \PY{n}{coefs}\PY{o}{.}\PY{n}{append}\PY{p}{(}\PY{n}{reg\PYZus{}lasso}\PY{o}{.}\PY{n}{coef\PYZus{}}\PY{p}{)}
\end{Verbatim}


    \begin{Verbatim}[commandchars=\\\{\}]
C:\textbackslash{}Users\textbackslash{}bandi\textbackslash{}Anaconda3\textbackslash{}lib\textbackslash{}site-packages\textbackslash{}sklearn\textbackslash{}linear\_model\textbackslash{}coordinate\_descent.py:491: ConvergenceWarning: Objective did not converge. You might want to increase the number of iterations. Fitting data with very small alpha may cause precision problems.
  ConvergenceWarning)

    \end{Verbatim}

    \begin{Verbatim}[commandchars=\\\{\}]
{\color{incolor}In [{\color{incolor}232}]:} \PY{n}{ax} \PY{o}{=} \PY{n}{plt}\PY{o}{.}\PY{n}{gca}\PY{p}{(}\PY{p}{)}
          \PY{n}{ax}\PY{o}{.}\PY{n}{plot}\PY{p}{(}\PY{n}{alphas}\PY{p}{,} \PY{n}{coefs}\PY{p}{)}
          \PY{n}{ax}\PY{o}{.}\PY{n}{set\PYZus{}xscale}\PY{p}{(}\PY{l+s+s1}{\PYZsq{}}\PY{l+s+s1}{log}\PY{l+s+s1}{\PYZsq{}}\PY{p}{)}
          \PY{n}{plt}\PY{o}{.}\PY{n}{xlabel}\PY{p}{(}\PY{l+s+s1}{\PYZsq{}}\PY{l+s+s1}{log lambda}\PY{l+s+s1}{\PYZsq{}}\PY{p}{)}
          \PY{n}{plt}\PY{o}{.}\PY{n}{ylabel}\PY{p}{(}\PY{l+s+s1}{\PYZsq{}}\PY{l+s+s1}{Coefficients}\PY{l+s+s1}{\PYZsq{}}\PY{p}{)}
          \PY{n}{plt}\PY{o}{.}\PY{n}{title}\PY{p}{(}\PY{l+s+s1}{\PYZsq{}}\PY{l+s+s1}{Lasso coefficients as a function of regularization}\PY{l+s+s1}{\PYZsq{}}\PY{p}{)}
\end{Verbatim}


\begin{Verbatim}[commandchars=\\\{\}]
{\color{outcolor}Out[{\color{outcolor}232}]:} Text(0.5,1,'Lasso Coefficients as a function of regularization')
\end{Verbatim}
            
    \begin{center}
    \adjustimage{max size={0.9\linewidth}{0.9\paperheight}}{output_26_1.png}
    \end{center}
    { \hspace*{\fill} \\}
    
    \begin{Verbatim}[commandchars=\\\{\}]
{\color{incolor}In [{\color{incolor}234}]:} \PY{c+c1}{\PYZsh{} Running Lasso Regression for all the alpha values}
          \PY{n}{alphas} \PY{o}{=}  \PY{l+m+mi}{10}\PY{o}{*}\PY{o}{*}\PY{n}{np}\PY{o}{.}\PY{n}{linspace}\PY{p}{(}\PY{l+m+mi}{10}\PY{p}{,}\PY{o}{\PYZhy{}}\PY{l+m+mi}{6}\PY{p}{,}\PY{l+m+mi}{100}\PY{p}{)}\PY{o}{*}\PY{l+m+mf}{0.5}
          \PY{n}{coefs} \PY{o}{=} \PY{p}{[}\PY{p}{]}
          \PY{k}{for} \PY{n}{i} \PY{o+ow}{in} \PY{n+nb}{range}\PY{p}{(}\PY{n+nb}{len}\PY{p}{(}\PY{n}{alphas}\PY{p}{)}\PY{p}{)}\PY{p}{:}
                  \PY{n}{reg\PYZus{}ridge} \PY{o}{=} \PY{n}{linear\PYZus{}model}\PY{o}{.}\PY{n}{Ridge}\PY{p}{(}\PY{n}{alpha} \PY{o}{=} \PY{n}{alphas}\PY{p}{[}\PY{n}{i}\PY{p}{]}\PY{p}{)}
                  \PY{n}{reg\PYZus{}ridge}\PY{o}{.}\PY{n}{fit}\PY{p}{(}\PY{n}{X\PYZus{}train}\PY{p}{,}\PY{n}{log\PYZus{}Y\PYZus{}train}\PY{p}{)}
                  \PY{n}{prediction} \PY{o}{=} \PY{n}{np}\PY{o}{.}\PY{n}{exp}\PY{p}{(}\PY{n}{reg\PYZus{}ridge}\PY{o}{.}\PY{n}{predict}\PY{p}{(}\PY{n}{X\PYZus{}test}\PY{p}{)}\PY{p}{)}
                  \PY{n}{coefs}\PY{o}{.}\PY{n}{append}\PY{p}{(}\PY{n}{reg\PYZus{}ridge}\PY{o}{.}\PY{n}{coef\PYZus{}}\PY{p}{)}
                  
          \PY{c+c1}{\PYZsh{} Plot ridge coefficients asa function of lambda}
          \PY{n}{ax} \PY{o}{=} \PY{n}{plt}\PY{o}{.}\PY{n}{gca}\PY{p}{(}\PY{p}{)}
          \PY{n}{ax}\PY{o}{.}\PY{n}{plot}\PY{p}{(}\PY{n}{alphas}\PY{p}{,} \PY{n}{coefs}\PY{p}{)}
          \PY{n}{ax}\PY{o}{.}\PY{n}{set\PYZus{}xscale}\PY{p}{(}\PY{l+s+s1}{\PYZsq{}}\PY{l+s+s1}{log}\PY{l+s+s1}{\PYZsq{}}\PY{p}{)}
          \PY{n}{plt}\PY{o}{.}\PY{n}{xlabel}\PY{p}{(}\PY{l+s+s1}{\PYZsq{}}\PY{l+s+s1}{log lambda}\PY{l+s+s1}{\PYZsq{}}\PY{p}{)}
          \PY{n}{plt}\PY{o}{.}\PY{n}{ylabel}\PY{p}{(}\PY{l+s+s1}{\PYZsq{}}\PY{l+s+s1}{Coefficients}\PY{l+s+s1}{\PYZsq{}}\PY{p}{)}
          \PY{n}{plt}\PY{o}{.}\PY{n}{title}\PY{p}{(}\PY{l+s+s1}{\PYZsq{}}\PY{l+s+s1}{Ridge coefficients as a function of regularization}\PY{l+s+s1}{\PYZsq{}}\PY{p}{)}
\end{Verbatim}


\begin{Verbatim}[commandchars=\\\{\}]
{\color{outcolor}Out[{\color{outcolor}234}]:} Text(0.5,1,'Ridge coefficients as a function of regularization')
\end{Verbatim}
            
    \begin{center}
    \adjustimage{max size={0.9\linewidth}{0.9\paperheight}}{output_27_1.png}
    \end{center}
    { \hspace*{\fill} \\}
    
    Overall as we increase the value of regularization term (lambda), the
coefficents become closer to zero.

    \begin{enumerate}
\def\labelenumi{\arabic{enumi})}
\setcounter{enumi}{2}
\tightlist
\item
  Run least squares regression, ridge, and lasso on the training data.
  For ridge and lasso, use only the best regularization parameter.
  Report the prediction error (MAE) on the test data for each. (5pts)
\end{enumerate}

    \begin{Verbatim}[commandchars=\\\{\}]
{\color{incolor}In [{\color{incolor}246}]:} \PY{c+c1}{\PYZsh{} Least Square Regression}
          \PY{n}{reg} \PY{o}{=} \PY{n}{linear\PYZus{}model}\PY{o}{.}\PY{n}{LinearRegression}\PY{p}{(}\PY{p}{)}
          \PY{n}{reg}\PY{o}{.}\PY{n}{fit} \PY{p}{(}\PY{n}{X\PYZus{}train}\PY{p}{,} \PY{n}{log\PYZus{}Y\PYZus{}train}\PY{p}{)}
          \PY{n}{prediction\PYZus{}test} \PY{o}{=} \PY{n}{np}\PY{o}{.}\PY{n}{exp}\PY{p}{(}\PY{n}{reg}\PY{o}{.}\PY{n}{predict}\PY{p}{(}\PY{n}{X\PYZus{}test}\PY{p}{)}\PY{p}{)}
          \PY{n}{MAE\PYZus{}test} \PY{o}{=} \PY{n}{mean\PYZus{}absolute\PYZus{}error}\PY{p}{(}\PY{n}{Y\PYZus{}test}\PY{p}{,}\PY{n}{prediction\PYZus{}test}\PY{p}{)}
          
          \PY{c+c1}{\PYZsh{} Lasso Regression}
          \PY{n}{reg\PYZus{}lasso} \PY{o}{=} \PY{n}{linear\PYZus{}model}\PY{o}{.}\PY{n}{Lasso}\PY{p}{(}\PY{n}{alpha} \PY{o}{=} \PY{n}{lambda\PYZus{}min\PYZus{}lasso}\PY{p}{)}
          \PY{n}{reg\PYZus{}lasso}\PY{o}{.}\PY{n}{fit}\PY{p}{(}\PY{n}{X\PYZus{}train}\PY{p}{,}\PY{n}{log\PYZus{}Y\PYZus{}train}\PY{p}{)}
          \PY{n}{prediction\PYZus{}lasso} \PY{o}{=} \PY{n}{np}\PY{o}{.}\PY{n}{exp}\PY{p}{(}\PY{n}{reg\PYZus{}lasso}\PY{o}{.}\PY{n}{predict}\PY{p}{(}\PY{n}{X\PYZus{}test}\PY{p}{)}\PY{p}{)}
          \PY{n}{MAE\PYZus{}test\PYZus{}lasso} \PY{o}{=} \PY{n}{mean\PYZus{}absolute\PYZus{}error}\PY{p}{(}\PY{n}{Y\PYZus{}test}\PY{p}{,}\PY{n}{prediction\PYZus{}lasso}\PY{p}{)}
          
          \PY{c+c1}{\PYZsh{} Ridge Regression}
          \PY{n}{reg\PYZus{}ridge} \PY{o}{=} \PY{n}{linear\PYZus{}model}\PY{o}{.}\PY{n}{Ridge}\PY{p}{(}\PY{n}{alpha} \PY{o}{=} \PY{n}{lambda\PYZus{}min\PYZus{}ridge}\PY{p}{)}
          \PY{n}{reg\PYZus{}ridge}\PY{o}{.}\PY{n}{fit}\PY{p}{(}\PY{n}{X\PYZus{}train}\PY{p}{,}\PY{n}{log\PYZus{}Y\PYZus{}train}\PY{p}{)}
          \PY{n}{prediction\PYZus{}ridge} \PY{o}{=} \PY{n}{np}\PY{o}{.}\PY{n}{exp}\PY{p}{(}\PY{n}{reg\PYZus{}ridge}\PY{o}{.}\PY{n}{predict}\PY{p}{(}\PY{n}{X\PYZus{}test}\PY{p}{)}\PY{p}{)}
          \PY{n}{MAE\PYZus{}test\PYZus{}ridge} \PY{o}{=} \PY{n}{mean\PYZus{}absolute\PYZus{}error}\PY{p}{(}\PY{n}{Y\PYZus{}test}\PY{p}{,}\PY{n}{prediction\PYZus{}ridge}\PY{p}{)}
          
          \PY{n+nb}{print}\PY{p}{(}\PY{l+s+s1}{\PYZsq{}}\PY{l+s+s1}{MAE for least square regression is:}\PY{l+s+s1}{\PYZsq{}}\PY{p}{,} \PY{n}{MAE\PYZus{}test}\PY{p}{)}
          \PY{n+nb}{print}\PY{p}{(}\PY{l+s+s1}{\PYZsq{}}\PY{l+s+s1}{MAE for lasso regression is:}\PY{l+s+s1}{\PYZsq{}}\PY{p}{,} \PY{n}{MAE\PYZus{}test\PYZus{}lasso}\PY{p}{)}
          \PY{n+nb}{print}\PY{p}{(}\PY{l+s+s1}{\PYZsq{}}\PY{l+s+s1}{MAE for ridge regression is:}\PY{l+s+s1}{\PYZsq{}}\PY{p}{,} \PY{n}{MAE\PYZus{}test\PYZus{}ridge}\PY{p}{)}
\end{Verbatim}


    \begin{Verbatim}[commandchars=\\\{\}]
MAE for least square regression is: 1810.6519682594767
MAE for lasso regression is: 1818.4224556783959
MAE for ridge regression is: 1818.8227140091647

    \end{Verbatim}

    \begin{enumerate}
\def\labelenumi{\arabic{enumi})}
\setcounter{enumi}{3}
\item
  Run lasso again with cross validation using
  \href{http://scikit-learn.org/stable/modules/generated/sklearn.linear_model.LassoCV.html}{sklearn.linear\_model.LassoCV}.
  Set the cross validation parameters as follows:

  LassoCV(alphas=None, cv=10, max\_iter=10000)
\end{enumerate}

Report the best \(\lambda\) based on cross validation. Run lasso on the
training data using the best \(\lambda\) and report the coefficeints for
all variables. (5pts)

    \begin{Verbatim}[commandchars=\\\{\}]
{\color{incolor}In [{\color{incolor}247}]:} \PY{c+c1}{\PYZsh{} Running Lasso with cross validation }
          \PY{k+kn}{from} \PY{n+nn}{sklearn}\PY{n+nn}{.}\PY{n+nn}{linear\PYZus{}model} \PY{k}{import} \PY{n}{LassoCV}
          \PY{n}{lasso\PYZus{}cv} \PY{o}{=} \PY{n}{LassoCV}\PY{p}{(}\PY{n}{alphas}\PY{o}{=}\PY{k+kc}{None}\PY{p}{,} \PY{n}{cv}\PY{o}{=}\PY{l+m+mi}{10}\PY{p}{,} \PY{n}{max\PYZus{}iter}\PY{o}{=}\PY{l+m+mi}{10000}\PY{p}{)}
          \PY{n}{lasso\PYZus{}cv}\PY{o}{.}\PY{n}{fit}\PY{p}{(}\PY{n}{X\PYZus{}train}\PY{p}{,} \PY{n}{log\PYZus{}Y\PYZus{}train}\PY{p}{)}
\end{Verbatim}


\begin{Verbatim}[commandchars=\\\{\}]
{\color{outcolor}Out[{\color{outcolor}247}]:} LassoCV(alphas=None, copy\_X=True, cv=10, eps=0.001, fit\_intercept=True,
              max\_iter=10000, n\_alphas=100, n\_jobs=1, normalize=False,
              positive=False, precompute='auto', random\_state=None,
              selection='cyclic', tol=0.0001, verbose=False)
\end{Verbatim}
            
    \begin{Verbatim}[commandchars=\\\{\}]
{\color{incolor}In [{\color{incolor}275}]:} \PY{n}{mse} \PY{o}{=} \PY{p}{[}\PY{p}{]}
          \PY{k}{for} \PY{n}{i} \PY{o+ow}{in} \PY{n+nb}{range}\PY{p}{(}\PY{n+nb}{len}\PY{p}{(}\PY{n}{lasso\PYZus{}cv}\PY{o}{.}\PY{n}{mse\PYZus{}path\PYZus{}}\PY{p}{)}\PY{p}{)}\PY{p}{:}
              \PY{n}{avg\PYZus{}mse} \PY{o}{=} \PY{n}{lasso\PYZus{}cv}\PY{o}{.}\PY{n}{mse\PYZus{}path\PYZus{}}\PY{p}{[}\PY{n}{i}\PY{p}{]}\PY{o}{.}\PY{n}{mean}\PY{p}{(}\PY{p}{)}
              \PY{n}{mse}\PY{o}{.}\PY{n}{append}\PY{p}{(}\PY{n}{avg\PYZus{}mse}\PY{p}{)}
              
          \PY{c+c1}{\PYZsh{} Calculate the lambda which minimizes mean squared error}
          \PY{n}{min\PYZus{}lambda\PYZus{}lasso\PYZus{}cv} \PY{o}{=} \PY{n}{lasso\PYZus{}cv}\PY{o}{.}\PY{n}{alphas\PYZus{}}\PY{p}{[}\PY{n}{mse}\PY{o}{.}\PY{n}{index}\PY{p}{(}\PY{n+nb}{min}\PY{p}{(}\PY{n}{mse}\PY{p}{)}\PY{p}{)}\PY{p}{]}
          
          \PY{c+c1}{\PYZsh{} Run the lasso regression on the minimum lambda}
          \PY{n}{reg\PYZus{}lasso} \PY{o}{=} \PY{n}{linear\PYZus{}model}\PY{o}{.}\PY{n}{Lasso}\PY{p}{(}\PY{n}{alpha} \PY{o}{=} \PY{n}{min\PYZus{}lambda\PYZus{}lasso\PYZus{}cv}\PY{p}{)}
          \PY{n}{reg\PYZus{}lasso}\PY{o}{.}\PY{n}{fit}\PY{p}{(}\PY{n}{X\PYZus{}train}\PY{p}{,}\PY{n}{log\PYZus{}Y\PYZus{}train}\PY{p}{)}
          
          \PY{c+c1}{\PYZsh{} Calculate the coefficients for all the variables}
          \PY{n}{reg\PYZus{}lasso}\PY{o}{.}\PY{n}{coef\PYZus{}}
\end{Verbatim}


\begin{Verbatim}[commandchars=\\\{\}]
{\color{outcolor}Out[{\color{outcolor}275}]:} array([-0.51711206,  0.35693216,  0.01261526, -0.08068485, -0.00950437,
                 -0.06881325,  0.30609698,  0.08840324,  0.57216477, -0.16869083,
                 -0.35348216,  0.55420528, -0.09951851,  0.09436384])
\end{Verbatim}
            
    \section{Question 6 (10pts)}\label{question-6-10pts}

Please solve problem \#3.3 on page 174 in Bishop (Chapter 3). The
problem has been uploaded to Canvas under 'Files': \textbf{CH3 problems
from Bishop PRML.pdf}.

\subsection{Answer}\label{answer}

    \begin{Verbatim}[commandchars=\\\{\}]
{\color{incolor}In [{\color{incolor}12}]:} \PY{n}{Image}\PY{o}{.}\PY{n}{open}\PY{p}{(}\PY{l+s+s2}{\PYZdq{}}\PY{l+s+s2}{Q6\PYZus{}1.jpg}\PY{l+s+s2}{\PYZdq{}}\PY{p}{)}
\end{Verbatim}


    \begin{Verbatim}[commandchars=\\\{\}]
Populating the interactive namespace from numpy and matplotlib

    \end{Verbatim}
\texttt{\color{outcolor}Out[{\color{outcolor}12}]:}
    
    \begin{center}
    \adjustimage{max size={0.9\linewidth}{0.9\paperheight}}{output_35_1.png}
    \end{center}
    { \hspace*{\fill} \\}
    

    \begin{Verbatim}[commandchars=\\\{\}]
{\color{incolor}In [{\color{incolor}13}]:} \PY{n}{Image}\PY{o}{.}\PY{n}{open}\PY{p}{(}\PY{l+s+s2}{\PYZdq{}}\PY{l+s+s2}{Q6\PYZus{}2.jpg}\PY{l+s+s2}{\PYZdq{}}\PY{p}{)}
\end{Verbatim}

\texttt{\color{outcolor}Out[{\color{outcolor}13}]:}
    
    \begin{center}
    \adjustimage{max size={0.9\linewidth}{0.9\paperheight}}{output_36_0.png}
    \end{center}
    { \hspace*{\fill} \\}
    

    Explanation:

\begin{enumerate}
\def\labelenumi{\arabic{enumi}.}
\item
  It might sometimes be important to strongly predict the response for
  certain values of inputs (maybe the ones which we might see often or
  maybe the ones where mistakes could be very costly). Thus if we give
  these values more weights and points elesewhere lesser weights, the
  regression will be pulled towards matching the data near those points.
\item
  \textbf{\emph{Data dependent noise variance}}: While linear regression
  assumes that noise in the data has constant variance, it is usually
  not correct in real data sets.If we know the noise variance of each
  measurement in the data set, we can set our weights in such a way
  which can minimize this variance.
\item
  \textbf{\emph{Replicated data points}}: It might be possible that some
  of the data points are over-sampled or have a high probability of
  including, while otherse could be under-sAmpled or have a low
  probability of including in the sample. We can then weight the data
  points in an inverse proportion to their probability of inclusion.
\end{enumerate}


    % Add a bibliography block to the postdoc
    
    
    
    \end{document}
